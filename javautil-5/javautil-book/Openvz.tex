\chapter{OpenVz}
\section{Installation}
http://openvz.org/Main_Page
\section{101}
I have created container 101 for wikidb
\begin{verbatim}
 


venet0:0  Link encap:UNSPEC  HWaddr 00-00-00-00-00-00-00-00-00-00-00-00-00-00-00-00
          inet addr:192.168.1.80  P-t-P:192.168.1.80  Bcast:192.168.1.80  Mask:255.255.255.255
          UP BROADCAST POINTOPOINT RUNNING NOARP  MTU:1500  Metric:1
\end{verbatim}

http://192.168.1.80

http://192.168.1.80/wiki/

http://www.yolinux.com/TUTORIALS/LinuxTutorialIptablesNetworkGateway.html

iptables -t nat -A POSTROUTING -o wlan0 -j MASQUERADE
http://www.tldp.org/HOWTO/IPCHAINS-HOWTO-7.html

[root@localhost home]# vzctl enter 101
entered into CT 101
[root@localhost /]# service mysqld start
Starting MySQL:                                            [  OK  ]




\section{Create a container}
\begin{verbatim}
 

\end{verbatim}


\begin{verbatim}
ROOT_MYSQL_PASSWORD='topSecret'     
LOGFILE=`pwd`/install.log           
echo_do() {                         
        echo $* | tee -a $LOGFILE   
        $* 2>&1 | tee -a $LOGFILE   
}                                   

test_echo() {
        echo_do echo this is a test
}                                  

install_software() {
        echo_do yum -y install php
        echo_do yum -y install apache
        echo_do yum -y install mysql 
        echo_do yum -y install rcs   
        echo_do yum -y install mysql-server 
        echo_do yum -y install php-mysql    
        echo_do yum -y install libapache2-mod-php5
}                                                 

get_media_wiki() {
        APACHE_ROOT=/var/www/html 
        MEDIA_WIKI=mediawiki-1.16.2
        cd $APACHE_ROOT            

        if [ $?  != 0 ] ; then
                echo unable to change to $APACHE_ROOT
                exit 1                               
        fi                                           

        if [ -d wiki ] ; then
                echo wiki already exists in $APACHE_ROOT
                exit 1                                  
        fi                                              

        if [ ! -f ${MEDIA_WIKI}.tar.gz ] ; then
                echo_do wget http://download.wikimedia.org/mediawiki/1.16/${MEDIA_WIKI}.tar.gz
        fi                                                                                    
        echo_do tar xfz ${MEDIA_WIKI}.tar.gz                                                  
        echo_do mv $MEDIA_WIKI wiki                                                           
}                                                                                             

configure_wiki() {
        # This is nasty but temporary
        echo_do chmod 777 /var/www/html/wiki/config
}                                                  

configure_mysql() {
        echo_do service mysqld start
        echo_do mysqladmin create wikidb
        echo_do mysqladmin -u root password $ROOT_MYSQL_PASSWORD 
}                                                                

configure_apache() {
        APACHE_CONF_DIR=/etc/httpd/conf
        FILE=${APACHE_CONF_DIR}/httpd.conf
        cd $APACHE_CONF_DIR               
        if [ 0 != 0 ] ; then              
                echo unable to change to  
        fi                                
        echo_do ci -l $FILE < /dev/null
        sed -i $FILE -e "/^[ *#]DirectoryIndex/a/ index.php/"
        LINENO=`grep -n LoadModule $FILE | tail -n 1 | cut -f 1 -d :`
        echo LINENO $LINENO
        COMMAND="$LINENO a LoadModule php5_module modules/libphp5.so"
        echo COMMAND $COMMAND
        sed -i -e "$COMMAND" $FILE
        service httpd stop
        service httpd start
}

run_conf() {
        echo connect to http://thishost/wiki/index.php and fill in the form
        echo after that succeeeds proceed to step two
}
test_echo
install_software
get_media_wiki
configure_wiki
configure_mysql
configure_apache
\end{verbatim}

\section{Create an Operating System Template}
http://openvz.org/Creating_a_CentOS_6_Template
\begin{verbatim}
 

\end{verbatim}

\section{Create a container}

\section{Install Nexus}
\subsection{Download}
http://www.sonatype.org/nexus/go

