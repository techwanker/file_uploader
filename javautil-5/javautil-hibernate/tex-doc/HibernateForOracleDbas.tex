
\maketitle

\section{Todo}

\begin{itemize}
 \item Need to get all of the events, waits and sql from v\$session to see what is truly going on.
 \item Get correct Sequence generation Strategy 
 \item Get the ant file to use the hbm.xml when generating the pojos, not doing it directly
 \item Show how to use the ejb style
 \item show how to get a native oracle connection and process arrays for fetching
 \item show how to customize the OracleDatasource and how it can be tuned 
 \item show how to generate and view Javadoc in eclipse
 \item create a generate javadoc task
 \item exclude from cvs 
 \item get column comments in the javadoc
 \item after you add a foreign key a reference to the object 
 \item show various ways to implement persistence including delimited files
 \item explain the difference between an interface and a package specification, multiple implementations
\end{itemize}



Hibernate is a widely used Object Relational Mapping tool, used to facilitate interaction with a SQL database from Java.

During the 2009 Hotsos Oracle Performance meeting it was widely accepted that there are many performance problems associated
with Hibernate but that ``nothing could be done about it''.

This article intends to dispel that myth with a brief description of the configuration of Hibernate and some small working
examples that demonstrate the points made in this article.  This is not a tutorial on Java or Hibernate.  It is expected that
these working examples will function as a prototype for in house use when warranted.  

A 99 percent improvement performance on something that burns a few CPU seconds a day and takes weeks to get into production is 
hardly cost effective.   Frequently run queries that burn large amounts of computer resources and create contention for Oracle 
serialized (latched, locked, enqueued, pinned...) resources are candidates for modification.
.
Just because something is sub-optimal does not mean that it should be improved.
The cost of ``implementing'' an improvement must be less than the value of the benefit derived from the improvement.  



Note that the default behavior of Object representation of  foreign keys are to objects of the type being referenced, which can be goofy.



Reverse Engineering,

Get the correct sequence generator in place



In or
\subsection{Install}
Create a working environment
\begin{itemize}

\item install the Operating System (centos / kubuntu / ??????)

\item install Oracle RDBMS
\item install Oracle EM
\item install Apex
\item Install Eclipse
\item Install SQLdeveloper
\item Install Apache
\item Install Tomcat

\end{itemize}



\section{Unit tests}







\section{todo}
Why Datasources.xml

\section{Introduction}
I intend to show various ways to improve Oracle performance when using Hibernate.  The java developers may not like all of these
techniques.   Hibernate does a good job when used as intended... (todo elaborate) 

It's payback time.  

DBA: Your execution plan is seriously suboptimal due to latch contention on busy block buffers due to consistent read on the application 
control table accross all of these sessions.

Java: I could read and cache the application control table entries and associate them with a HashMap to my detail entries and write a strategy
to control my secondary cache in attempt to keep in sync with committed transactions but the change notification mechanism doesn't ship until
Oracle 11 which sure isn't ready for production use yet.  Seems like you're putting the work of database management on the app side and we are
re-inventing the wheel.  Now I could create a AOP cut point to send a JMS ....


todo transaction demarcation 

\subsection{Terminology}
Child Relation - Frequently referred to as a detail table, this the ``many end'' of a ``one to many'' relationship.  

\subsection{Hibernate Strengths}
todo flesh out
\subsection{Hibernate Weakness}
To properly 


\begin{enumerate}
 \item Why you should create constraints on views if you are using Hibernate Reverse Engineering.
 \item Describe the various fetch strategies and how they can be modified
 \item Describe a mechanism to reduce hard parses. 
 \item Key generation
 \item Using Views with Relations   
 
\end{enumerate}

\subsection{Improving Performance}
With a Session Managed Bean in hand 
a traversal to a \textit{child relation} Hibernate may find it appropriate 
to fetch all of the todo associated with the other \textit{Parent relation} records.




\subsection{More Tips}



\section{Fetching Policies}
\subsection{Eager Fetch}
An eager fetch assumes that the child tables of a given table will be visited and so will be fetched by Hibernate in advance of
any attempt to reference the child relation.


\subsection{Lazy Fetch}


\section{Declarative Fetch Policy}

\begin{itemize}
 \item Eager fetching.
\end{itemize}

\section{Lets Get On With Tutorial}
\subsection{Create The Schema}
I assume that the dba creates the schema and that the java developers reverse engineer the schema into hibernate.

Let's assume a schema \ref{CreateTheSchema} on \pageref{CreateTheSchema}

\subsection{Create a Reverse Engineering Strategy}

\subsection{Generate the mapping files}

\subsection{  }
Now Fetch the masters, create an ID list and fetch the details followed by a mapped association of details to master.


\label{CreateTheSchema}
% todo create a link to sales.sql
%! \verbatiminput{../../ddl/bank.sql} 



\end{document}
