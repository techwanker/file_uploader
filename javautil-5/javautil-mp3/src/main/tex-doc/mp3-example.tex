\documentclass[a4paper,10pt]{book}
\usepackage[latin1]{inputenc}
\usepackage{amsfonts}
\usepackage[T1]{fontenc}
\usepackage[dvips]{graphicx}
\usepackage{alltt} 
\usepackage{moreverb} 
 \usepackage[top=2cm, bottom=2cm, left=1.5cm, right=1.5cm]{geometry} 
\usepackage[bookmarks=true]{hyperref}
\usepackage{underscore}
\usepackage{boxedminipage}
\usepackage{float}
\usepackage{hyperref}

%opening
\title{JDBC utilities}
\author{jjs@javautil.org}
\hypersetup{
pdfauthor = {jjs@javutil.org},
pdftitle = {Whirlwind tour of Spring, Hibernate and Maven},
pdfsubject = {A quick introduction to Spring, Hibernate and Maven},
pdfkeywords = {JDBC},
pdfcreator = {LaTeX with hyperref package},
pdfproducer = {dvips + ps2pdf}}
% Alter some LaTeX defaults for better treatment of figures:
%http://mintaka.sdsu.edu/GF/bibliog/latex/floats.html
    % See p.105 of "TeX Unbound" for suggested values.
    % See pp. 199-200 of Lamport's "LaTeX" book for details.
    %   General parameters, for ALL pages:
    %\renewcommand{\topfraction}{0.9}	% max fraction of floats at top
    %\renewcommand{\bottomfraction}{0.8}	% max fraction of floats at bottom
    %   Parameters for TEXT pages (not float pages):
    %\setcounter{topnumber}{2}
    %\setcounter{bottomnumber}{2}
    %\setcounter{totalnumber}{4}     % 2 may work better
    %\setcounter{dbltopnumber}{2}    % for 2-column pages
    %\renewcommand{\dbltopfraction}{0.9}	% fit big float above 2-col. text
    %\renewcommand{\textfraction}{0.07}	% allow minimal text w. figs
    %   Parameters for FLOAT pages (not text pages):
    %\renewcommand{\floatpagefraction}{0.7}	% require fuller float pages
	% N.B.: floatpagefraction MUST be less than topfraction !!
    %\renewcommand{\dblfloatpagefraction}{0.7}	% require fuller float pages

	% remember to use [htp] or [htpb] for placement
\floatstyle{ruled}
\newfloat{program}{thp}{lop}
\floatname{program}{Program}
\setlength{\parindent}{0pt}  % don't indent paragraphs?
\setlength{\parskip}{1ex plus 0.5ex minus 0.2ex} % inter paragraph spacing


\begin{document}
\chapter{Artist Resolution}
Let us start with the simple assertion ``An artist is defined by what I want it to be''

After examing hundreds of thousands of mp3s we find that artists are defined by their name.

As there is not a standards agency that doles out band identifiers. We must derive one.

I start with the assertion that each ``band'' must be assigned an immutable arbitrarily assigned identifier.

\begin{itemize}
 \item Beatles
 \item The Beatles
 \item Beatles, The  
\end{itemize}

What is the ``Name by which you would like to have The Beatles identified?''

Simple is good.

Let us define a table and call it artist.

The unique identifier for artist shall be ``artist_id''.

This identifier should be immutable, that is should never change.

A property of artist is the preferred name for the artist.

Natural keys.

\begin{tabular}{ l | l | l }
artist_id & artist identifier & immutable means of specifying an artist \\
artist_name & The preferred user friendly representation of an artist's name 
 
\end{tabular} 

The artist will be referred to in relational objects by the identifier.

Now we need some way of assigning an artist_id to various representations of artists names.


todo with various singers?
\begin{verbatim}
select artist_name, count(*) 
from mp3 
where upper(artist_name) like '%BEATLES%' 
group by upper(artist_name)
\end{verbatim}


\begin{tabular}{l | l}
Artist Name	& Count \\ 
Beatles, The	& 28 \\
Beatles	 & 81 \\
 The Beatles	& 2 \\
Beatlesles	& 1 \\
The Beatles	& 559
\end{tabular} 

Popularity is a good indicator of preference. 
Based on the above one might be inclined to name the band ``The Beatles''.  

should be able to group
"Allman Brothers",6
"Allman Brothers Band",36
"Allman Brothers Band, The",1
"Allmon Brothers Band",1
"Alman Brothers Band",1

"Beatles",31
"Beatles, The",16
"Beatlesles",1
"The Beatles",34

"Blood Sweat & Tears",1
"Blood Sweat And Tears",1
"Blood Sweat and Tears",1
"Blood, Sweat & Tears",5
"Blood, Sweat And Tears",2
"Blood, Sweat, And Tears",2

"Christina Aguilera",14
"Christina Agularia",1


should be able 

\subsection{Word Extractor}
TODO describe
package org.javautil.mp3;

public interface WordExtractor {
	
	public String[] getWords(String text);
}

don't get carried away, don't write an interface and require injection of everything you write.

comment on AutoWired why it is bad.

http://www.javacodegeeks.com/2011/01/10-tips-proper-application-logging.html

testing each test should stand alone, not be dependent on prior tests or ordering

http://code.google.com/p/google-diff-match-patch/

We can pair every artist name with every other artist name that comes close.

Now how do we create a group?  What will be the group identifier

csv tokenizers and readers

\subsection{Sort Name}
processor chains 

Band names should upper case 
'The' and '.*The' should be stripped


notes 

in

todo explain insertCountBetweenCommits

null empty strings

Review and take actions  don't try to write too much code, too many exceptions.

todo create a separate file for Session Factory and use with other files in spring configuration

make sure .hbm.xml files are in right place get sllly messages
\end{document}
