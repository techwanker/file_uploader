\documentclass[a4paper,10pt]{book}
\usepackage[latin1]{inputenc}
\usepackage{amsfonts}
\usepackage[T1]{fontenc}
\usepackage[dvips]{graphicx}
\usepackage{alltt} 
\usepackage{moreverb} 
 \usepackage[top=2cm, bottom=2cm, left=1.5cm, right=1.5cm]{geometry} 
\usepackage[bookmarks=true]{hyperref}
\usepackage{underscore}
\usepackage{boxedminipage}
\usepackage{float}
\usepackage{hyperref}

%opening
\title{JDBC utilities}
\author{jjs@javautil.org}
\hypersetup{
pdfauthor = {jjs@javutil.org},
pdftitle = {Whirlwind tour of Spring, Hibernate and Maven},
pdfsubject = {A quick introduction to Spring, Hibernate and Maven},
pdfkeywords = {JDBC},
pdfcreator = {LaTeX with hyperref package},
pdfproducer = {dvips + ps2pdf}}
% Alter some LaTeX defaults for better treatment of figures:
%http://mintaka.sdsu.edu/GF/bibliog/latex/floats.html
    % See p.105 of "TeX Unbound" for suggested values.
    % See pp. 199-200 of Lamport's "LaTeX" book for details.
    %   General parameters, for ALL pages:
    %\renewcommand{\topfraction}{0.9}	% max fraction of floats at top
    %\renewcommand{\bottomfraction}{0.8}	% max fraction of floats at bottom
    %   Parameters for TEXT pages (not float pages):
    %\setcounter{topnumber}{2}
    %\setcounter{bottomnumber}{2}
    %\setcounter{totalnumber}{4}     % 2 may work better
    %\setcounter{dbltopnumber}{2}    % for 2-column pages
    %\renewcommand{\dbltopfraction}{0.9}	% fit big float above 2-col. text
    %\renewcommand{\textfraction}{0.07}	% allow minimal text w. figs
    %   Parameters for FLOAT pages (not text pages):
    %\renewcommand{\floatpagefraction}{0.7}	% require fuller float pages
	% N.B.: floatpagefraction MUST be less than topfraction !!
    %\renewcommand{\dblfloatpagefraction}{0.7}	% require fuller float pages

	% remember to use [htp] or [htpb] for placement
\floatstyle{ruled}
\newfloat{program}{thp}{lop}
\floatname{program}{Program}
\setlength{\parindent}{0pt}  % don't indent paragraphs?
\setlength{\parskip}{1ex plus 0.5ex minus 0.2ex} % inter paragraph spacing


\begin{document}
\title{Whirlwind tour of Spring, Hibernate and Maven}
\author{Jim Schmidt\\
  \texttt{jjs@javautil.org}}
\date{\today}
\maketitle
\tableofcontents
\part{I}
\chapter{Introduction}
You've just been assigned to a new project group that uses some of the most popular development libraries and 
methodologies.

This is a series of lessons designed to quickly give you an introduction to Maven on page Maven , Spring and Hibernate.

During the course of this the project specifications will be continually revised.

The author of this book found code snippets but no examples of real applications that used these practices.

What are the practices we are going to cover?
\label{Maven}
\section{Maven}

\begin{quotation}
Apache Maven is a software project management and comprehension tool. 
Based on the concept of a project object model (POM), Maven can manage a project's build, 
reporting and documentation from a central piece of information.
\end{quotation}
\section{Hibernate}
\url{http://en.wikipedia.org/wiki/Hibernate_Java}
\begin{quotation}
Hibernate is an object-relational mapping (ORM) library for the Java language, 
providing a framework for mapping an object-oriented domain model to a traditional relational database. Hibernate solves object-relational impedance mismatch problems by replacing direct persistence-related database accesses with high-level object handling functions.

Hibernate is free software that is distributed under the GNU Lesser General Public License.

Hibernate's primary feature is mapping from Java classes to database tables (and from Java data types to SQL data types). 
Hibernate also provides data query and retrieval facilities. 
Hibernate generates the SQL calls and relieves the developer from manual result set handling and object conversion, 
keeping the application portable to all supported SQL databases, with database portability delivered at very little performance overhead. 
\end{quotation}
\section{Spring}
Spring is a framework... todo

\section{H2}
\url{http://www.h2database.com/html/main.html}
\begin{quotation}
 Welcome to H2, the Java SQL database. The main features of H2 are:

    * Very fast, open source, JDBC API
    * Embedded and server modes; in-memory databases
    * Browser based Console application
    * Small footprint: around 1 MB jar file size 
\end{quotation}
When you are done with this you should be able to build projects using the primary components of the above frameworks.  This is an introduction
to tying all of these technologies together.  After you are done you should be pretty excited and want to get some books that I going
to recommend and apply these to your projects.

I will show you best practices with respect to database design but not dwell on them.

The concepts in this application apply to a broad range of data acquisition projects.
\section{The Application}
Hopefully you'll find the application interesting enough that you will want to play with long after your done.  Tweak it, add some 
features.  Maybe you can contribute some back to the project. That's a big part of open source.

We will write a program that will read all of your MP3 files (or those in the directories you specify and write the information somewhere.

What is somewhere?  We write it to a relational database using jdbc, hibernate, java persistence api 2.0, comma separated values files, excel
spreadsheets, html files, pdfs, xml.  We will ``wire'' the application using Spring to use each of the difference persistence classes. The 
application is dependent on a large amount of open source code, but maven will do the work of downloading and installing those.

Having specified that the destination is a relational database will allow us to get right into the data model
that is to be created and circumvent the argument as to which comes first the domain model or the data model.

TODO what is the difference?
\chapter{Installation}

\section{Prerequisites}
Install java 1.6 or higher
Install maven


Create a workspace

\section{Download} 
download javautil-mp3-src.taz from \url{http://javautil.org/download/javautil-mp3-src.taz} 
\section{Extract}
Extract the file using winzip or on linux 
tar xvfz javautil-mp3-src.taz

TODO find example
TODO what about javautil-pom

javautil-core
javautil-commandline
javautil-pom
maven-artifacts

\section{Install objects not in repository}
Maven will locate artifacts from web resources

There are a few objects that are not in any online repositories at the time of this writing.  We will install them manually.

\begin{program}
 \verbatiminput{../../../../maven-artifacts/mvn-install-artifacts}
\end{program}
What just happened?

Well if everything worked, those jar files were put in a common location that maven uses.  In your home directory .m2/repository.
Under windows your home directory varies from version to version.  But this is nice, all the projects that use these files automatically know
where to look for them.  It makes them very easy to share.  This is one of the many examples of TODO over configuration.
Kind of a heavy handed ``do it my way and no one will get hurt'' Most of these standard places can be over-ridden but it is work and 
makes your project not look like the next guys.   Find the repository and browse around. Then read TODO THIS.

What are these jar files?
\begin{tabular}{ l | l | l | l | l}
Group Id & Artifact Id & Version & jarfile & Description \\
\hline
jcmdline  & jcmdline & 1.0.3 & jcmdline-1.0.3.jar & Java Command Line Processor\\
\hline
com.mchange & c3p0  & 0.9.1 & c3p0-0.9.1.jar  & A DataSource implementation\\
\hline
com.google.code.gson & gson & 1.3 & gson-1.3.jar & Google support for JSON \\
\end{tabular}

\section{Project Specs}
\subsection{Initial Request}
Design a program to scan a disk drive and store the contents of an MP3 files into an open ended number of formats 
and persistence mechanisms.

As with any data gathering exercise it is expected that ...

Need the ability to normalize and identify the artist.




\subsection{Support Multiple drives and systems}
\section{Command Line}
I love eclipse, but everything should be supportable from the command line.   If you want to use Maven in eclipse GO
TODO.  I frequently find troubles for programmers by trying mvn from the command line.  If that doesn't work I fix the offending problem.
If it does, I usually do a 
\begin{verbatim}
mvn eclipse:clean
mvn eclipse:eclipse 
\end{verbatim}
If you know eclipse, great, I couldn't code without it anymore, it is too configurable, my screenshots might not look like yours, it changes
from version to version and the screenshots would really bloat this document.  Besides, it is nice to know what is going on behind the scenes.
Java server apps are usually deployed on ``headless'' computers, with no graphics software installed, the less installed, the few conflicts and 
security holes.

\section{Build the code}
remove any old versions
mvn clean
compile
mvn install

Phases but the code was compiled and unit tests were run.

build a website for our project
mvn site

Browse around
firefox target/site/index.html

Pretty cool, you have javadoc. We ran a bunch of unit tests and we see that, hopefully everything passed,  you can see what code was tested what was not

\section{Systems Analyst}
A disk drive?  A variety of data persistent stores, the internet, the local area network

Domain model

create project 

in eclipse install maven plugin

enable dependence management (show screen shots)

We want to use the features available in version 1.5 of java

A lot of tabs, describe them



plugins

click add

need to add 

Now add the following property

    <junit.version>4.4</junit.version>
because the build tool defaulted to a different version.

Working from the command line so that we can isolate out eclipse configuration problems.

maven from the command line

mvn eclipse:clean
mvn eclipse:eclipse

configure in junit 4.4
 \begin{minipage}[b]{1.5in}
  Beans Cause Farts.
 \end{minipage}

Please read \url{http://maven.apache.org/pom.html#Maven_Coordinates}

\section{installing library}
We have decided to use jidlib now we must install it into our repository

Need to

run src/main/lib/install_jars_to_maven_repository.sh 

Need to change to use java 1.6 but unable to configure through gui in eclipse need to edit pom.

Create a bean for the MP3 metadata 

\section{Design}
Need a bean to hold the information from an MP3 file.
TODO need to preserve tabs
\verbatiminput{../java/org/javautil/mp3/MP3Metadata.java}
TODO want to test that it works correctly

test the class

\section{Create the database objects}
Talk about what comes first database objects or domain objects.

\section{Creating an argument Beans}
see javautil-commandline
alter the pom.

Cut and paste this into the pom.
\begin{verbatim}
   <dependency>
               <groupId>org.javautil</groupId>
               <artifactId>javautil-commandline</artifactId>
               <version>${org.javautil.commandline.version}</version>
    </dependency>
\end{verbatim}
Create the bean
Create the properties file, we will show you how to use to process command line arguments later.

configure a log4j.xml in src/test/resources and make sure it gets deployed.

How much have we tested?  Code coverage.

CSV's as test and expected results.

\section{Now what}
Now we have MP3MetaDataExtractorTest how can we configure this another way?

\section{Inject over factories}
public interface Mp3Persistence

How the persistence came into being is not a matter of concern of the application that wishes to store data.

Is it writing to a socket? A database? A file?  Why do I care?

Every one of these implementations require different constuctors 
and or setter methods before than can do their work

\section{write to flat file as }

\section{Implementations}
Document Object Model
Streaming XML
CSV
Excel Workbook
HTML
JASON
Serialized Java Object
JDBC
Hibernate
Service Call
Dataset
EBCDIC

\section{The canonical representation, the domain model}

\chapter{Miscellaneous}
\section{Logging}

\section{Private Methods}
\section{TODO}
show cascading inherited beans.

\section{JDBC}
now we add a dependency on javautil-jdbc to the project configuration, rebuild the eclipse project.

\section{Creating The Schema}
In the real world large amounts of data are stored in relational databases. 

Without support, based on my experience, scripts are used to create database objects in production
databases.  These scripts are reviewed by Database Administrators. 

We will be creating database objects programmatically for small apps this is fairly common also we
want to start fresh for our automated testing.

The data generally outlives the applications that created them. This is a fairly safe statement.

Generally use scripts, for the purpose of partitions etc.  Table comments, column comments.

Script runner
\section{HenPlus}
\section{H2}
What is H2?  From its website
\begin{quotation}
 Welcome to H2, the Java SQL database. The main features of H2 are:

    * Very fast, open source, JDBC API
    * Embedded and server modes; in-memory databases
    * Browser based Console application
    * Small footprint: around 1 MB jar file size 
Let's start up an H2 database and create some tables.
\end{quotation}
\section{Oracle}

\chapter{Hibernate}
\section{Reverse Engineering}
Reverse Engineering database tables consists of four files
\section{Ant}
build.xml
\section{hibernate.cfg.xml}
\section{hibernate.reveng.xml}
\url{http://docs.jboss.org/tools/2.1.0.Beta1/hibernatetools/html/reverseengineering.html}
\section{Strategy}



\section{Dialect} 
 org.hibernate.cfg.reveng.dialect.JDBCMetaDataDialect  contribute for oracle TODO

TODO sequences for primary keys

edit the build.xml file.

\section{Dependencies}
Build schema
Build Reverse engineer strategy
build maps

TODO now it gets ugly how do we define the location of all of the files necessary to make this work? Input into lib?  
Why can't we reference our maven repository.

Now the ant file has a dependency on the reverse strategy that has not yet been built.

After a clean the test database needs to be created which requires

does database username in hibernate.cfg.xml have to be in upper case for revenge?  Otherwise connect but don't find any objects.
\url{http://mojo.codehaus.org/maven-hibernate3/hibernate3-maven-plugin/usage.html}

The code in the repositories is different than the compiled code in Eclipse.

TODO get rid of /etc entries

building with passwords for the database in the project
Build the mapping files from the database
mvn hibernate3:hbm2hbmxml

hibernate3:hbm2cfgxml
  Generates hibernate.cfg.xml
\begin{tabular}{ l | l}
 
hibernate3:hbm2dao &
  Base class for the different hibernate3 goals based on the Ant tasks of
  hibernate tools. \\ \newline

hibernate3:hbm2ddl &
  Generates database schema \\.

hibernate3:hbm2doc &
  Generates HTML documentation for the database schema. \\

hibernate3:hbm2hbmxml &
  Generates a set of hbm.xml files \\

hibernate3:hbm2java &
  Generates Java classes from set of *.hbm.xml files  \\

hibernate3:hbmtemplate &  Generic exporter that can be controlled by a user provided template or class. \\

hibernate3:help
  Display help information on hibernate3-maven-plugin. Call
    mvn hibernate3:help -Ddetail=true -Dgoal=<goal-name>
  to display parameter details.

\end{tabular}

The package name should be specified as package
\begin{program}
 \begin{verbatim}
   <componentProperties>
            <drop>true</drop>
            <package>org.javautil.mp3.hibernate</package>
            <configurationfile>/src/main/resources/hibernate.cfg.xml</configurationfile>
   </componentProperties>
 \end{verbatim}
  \caption{The Hibernate Reverse Engineering}
\end{program}

 

Not TODO packagename as specified in the documentation.

failure to specify a package name will result in no package.

\url{http://stackoverflow.com/questions/1900234/maven-java-source-code-generation-for-hibernate}

\section{Transaction}
Track artists.
Musicians by song.
Found Name vs
\part{Hibernate}
\url{http://mojo.codehaus.org/maven-hibernate3/hibernate3-maven-plugin/componentproperties.html}

\begin{tabular}{l | l | l}
configurationfile & String & src/main/resources/hibernate.cfg.xml \\ 
\hline
propertyfile & String & src/main/resources/database.properties \\ 
\hline
entityresolver & String & org.xml.sax.EntityResolver \\ 
\hline
namingstrategy & String & org.hibernate.cfg.DefaultNamingStrategy \\
\hline
persistenceunit & String & \\ 
\hline
packagename & String & \\ 
\hline
revengfile & String & \\ 
\hline
reversestrategy & String & \\
\hline
detectmanytomany & boolean & true \\ 
\hline
detectoptmisticlock & boolean & true \\ 
\hline
export & boolean & true \\ 
\hline
update & boolean & false \\ 
\hline
drop & boolean & false \\ 
\hline
create & boolean & true \\ 
\hline
outputfilename & String & \\ 
\hline
delimiter & String & \\ 
\hline
format & boolean & false \\ 
\hline
jdk5 & boolean & false \\ 
\hline
ejb3 & boolean & false \\ 
\hline
filepattern & String & package-name/class-name.ftl \\ 
\hline
template & String & \\

basedir/
exporterclass	String	
scan-classes	boolean	false
\end{tabular} 

should one have a separate class that is an InitializingBean that extends a bean

InitializingBean afterPropertiesSet

eliminate Spring depenendencies
\part{Maven}

questions 

where does this go in the file  dbcommons in the pom.xml

test test tst

diagrams

code coverage testing

regression

consistent versions

transactional integrity, recovery restart
%http://www.andy-roberts.net/misc/latex/latextutorial6.html
\begin{figure}[htpb]
	\begin{boxedminipage}{2in}
		What is the point of maven?

	        What is a duck?
	\end{boxedminipage}
	%\cite{My Text Box}
	\label{box:mybox}
\end{figure}
\section{Scrolling Results}
\begin{verbatim}
 http://www.mastertheboss.com/jboss-howto/65-hibernate/220-how-to-open-an-online-cursor-with-hibernate.html

http://en.wikipedia.org/wiki/ID3
\end{verbatim}

\chapter{The data model}
Having asserted that the objective of this project is to populate a database we have bypassed the argument
of what drives the model, the domain model or the database model.

Hibernate will allow a data model to be created using java classes or mapping files but this is contrary
to a pattern familiar to a database administrator.  The best practice for database models is that the database 
objects are owned by one TODO schema and the rights to insert and update the model are granted to another 
database user.  Database objects are generally not created and modified on the fly in production by 
programmers.   

There are tools that allow a data modeller to graphically design a data model and then generate DDL
but most expert database administrators find it far superior to just write the DDL and reverse engineer
the diagrams, for a variety of reasons.  It is far faster, version control management is easier.

What is preserved.

What is not preserved
   Check Constraints


\section{The DDL}


\verbatiminput{../resources/create_tables.h2.sql}

Create

\section{Data Scrubbing}

\section{Operation}
\subsection{Data Aquisition}
Compressed Files of Data
Directory trees containing data
Comma Separated Files of data from other systems
\subsection{Initial Persistence}

\subsection{Data Scrubbing}
Update artist ID on mp3 for exact hits.
Populate Artist and Artist_Alias

\subsection(Operation)
Merge artist into artist
TODO show how to do in database, locking with transaction number.  Should this be called transaction number or action number.
TODO need the ability to identify error conditions on data using declarative and procedural techniques.

TODO log transaction details

\subsection{Posting}
\subsection{Extraction and Reporting}
Comma Separated Values
XML
Spread Sheets
\subsection{Web Services}

\subsection{Updating}

W


TODO remove cache provider from javautil-mp3 pom.xml and figure out the problem

TODO then have to add javassist

TODO JDBC Template example

Artist

TODO tune 

http://www.slf4j.org/faq.html

mvn dependency:tree

TODO
add dropIfExists options on tables , run for both h2 and

big fat jarfile

-DdownloadJavadocs=true

install maven plugin eclipse

install plugin to eclipse

mvn dependency:sources

performance monitoring under eclipse

configure a logger for errors

don't build database if it exists.

-----------
Extract artists
create artist to normalize

populate artist

add transaction number to various tables so we know who changed it and  why?

there is an id v2 band tag

eliminate need for outputFile have a different argument class for Csv extractor

exercise change the password on the database and the associated files (from build.xml etc )

Create character scrubber 

modify revenge strategy so that mp_seq becomes mp3_id_seq

don't have schema generated in hbm.xml and explain why

public static byte[] hexStringToByteArray(String s) {
    int len = s.length();
    byte[] data = new byte[len / 2];
    for (int i = 0; i < len; i += 2) {
        data[i / 2] = (byte) ((Character.digit(s.charAt(i), 16) << 4)
                             + Character.digit(s.charAt(i+1), 16));
    }
    return data;
}
Reasons why it is an improvement:

Safe with leading zeros (unlike BigInteger) and with negative byte values (unlike Byte.parseByte)

Doesn't convert the String into a char[], or create StringBuilder and String objects for every single byte.

Feel free to add argument checking via assert or exceptions if the argument is not known to be safe.

----------------------------------

Describe intermittent commits

Need the ability to specify the log file name on the command line

Need the ability to see output from the command runner.

Todo check encoding language 

support for transaction number to back track as to how data got some way.  Including service, version and rule name.

strip out non ascii characters and all control characters from data persisted in database.  It is difficult to type and display these characters.


create tables by running script runner

reverse engineer, where should the class files go and the mapping files

the beauty of naming conventions  Reverse Engineering.

TODO
   reverse engineering get the schema out of the name  
   match schema in hibernate.reveng.xml

https://forum.hibernate.org/viewtopic.php?f=6&t=995548&start=0/url

references

http://docs.jboss.org/tools/3.0.1.GA/en/hibernatetools/html_single/index.html#hibernaterevengxmlfile

unit testing 

functional testing

\chapter{Tests}
\section{Eclipse}



\end{document}
