\documentclass[a4paper,10pt]{book}

%opening
\title{JDBC utilities}
\author{jjs@javautil.org}

\begin{document}
\title{Whirlwind tour of Spring, Hibernate and Maven}
\author{Jim Schmidt\\
  \texttt{jjs@javautil.org}}
\date{\today}
\maketitle
\tableofcontents
\part{I}
\chapter{Introduction}
This book is a whirlwind tour of technologies that are .........

\begin{itemize}
 \item Java
 \item Relational Databases
 \item Spring (Dependency Injection)
 \item Hibernate
 \item Maven


\end{itemize}


You've just been assigned to a new project group that uses some of the most popular development libraries and 
methodologies.

This is a series of lessons designed to quickly give you an introduction to Maven, Spring and Hibernate.

During the course of this the project specifications will be continually revised.
\section{Project Specs}
\subsection{Initial Request}
Design a program to scan a disk drive and store the contents of an MP3 files into an open ended number of formats 
and persistence mechanisms.
\subsection{Support Multiple drives and systems}
\section{Command Line}
I love eclipse, but everything should be supportable from the command line.

\section{Systems Analyst}
A disk drive?  A variety of data persistent stores, the internet, the local area network
\section{Steps}
Download and install Maven.

Create your directories.

create new directory in Eclipse

Domain model

create project 

in eclipse install maven plugin

enable dependence management (show screen shots)

We want to use the features available in version 1.5 of java

A lot of tabs, describe them



plugins

click add

need to add 

Now add the following property

    <junit.version>4.4</junit.version>
because the build tool defaulted to a different version.

Working from the command line so that we can isolate out eclipse configuration problems.

maven from the command line

mvn eclipse:clean
mvn eclipse:eclipse

configure in junit 4.4

\section{installing library}
We have decided to use jidlib now we must install it into our repository

Need to


Need to change to use java 1.6 but unable to configure through gui in eclipse need to edit pom.

Create a bean for the MP3 metadata 

\section{Design}
Need a bean to hold the information from an MP3 file.
TODO need to preserve tabs

TODO want to test that it works correctly

test the class

\section{Create the database objects}
Talk about what comes first database objects or domain objects.

\section{Creating an argument Beans}
see javautil-commandline
alter the pom.

Cut and paste this into the pom.
\begin{verbatim}
   <dependency>
                        <groupId>org.javautil</groupId>
                        <artifactId>javautil-commandline</artifactId>
                        <version>${org.javautil.commandline.version}</version>
    </dependency>



\end{verbatim}
Create the bean
Create the properties file, we will show you how to use to process command line arguments later.

configure a log4j.xml in src/test/resources and make sure it gets deployed.

How much have we tested?  Code coverage.

CSV's as test and expected results.

\section{Now what}
Now we have MP3MetaDataExtractorTest how can we configure this another way?

\section{Inject over factories}
public interface Mp3Persistence

How the persistence came into being is not a matter of concern of the application that wishes to store data.

Is it writing to a socket? A database? A file?  Why do I care?

Every one of these implementations require different constuctors 
and or setter methods before than can do their work

\section{write to flat file as }

\section{Implementations}
Document Object Model
Streaming XML
CSV
Excel Workbook
HTML
JASON
Serialized Java Object
JDBC
Hibernate
Service Call
Dataset
EBCDIC

\section{The canonical representation, the domain model}

\chapter{Miscellaneous}
\section{Logging}

\section{Private Methods}
\section{TODO}
show cascading inherited beans.

\section{JDBC}
now we add a dependency on javautil-jdbc to the project configuration, rebuild the eclipse project.

\section{Creating The Schema}
In the real world large amounts of data are stored in relational databases. 

Without support, based on my experience, scripts are used to create database objects in production
databases.  These scripts are reviewed by Database Administrators.

The data generally outlives the applications that created them. This is a fairly safe statement.

Generally use scripts, for the purpose of partitions etc.  Table comments, column comments.

Script runner
\section{HenPlus}
\section{H2}
\section{Oracle}
\chapter{Reverse Engineering}
Reverse Engineering database tables consists of four files
\section{Ant}
build.xml
\section{hibernate.cfg.xml}
\section{hibernate.reveng.xml}
\begin{verbatim}
http://docs.jboss.org/tools/2.1.0.Beta1/hibernatetools/html/reverseengineering.html#custom-reveng-strategy
\end{verbatim}
\section{Strategy}
\begin{verbatim}
http://docs.jboss.org/tools/2.1.0.Beta1/hibernatetools/html/reverseengineering.html#custom-reveng-strategy
\end{verbatim}
\section{Dialect} 
 org.hibernate.cfg.reveng.dialect.JDBCMetaDataDialect  contribute for oracle TODO

TODO sequences for primary keys

edit the build.xml file.

\section{Dependencies}
Build schema
Build Reverse engineer strategy
build maps

TODO now it gets ugly how do we define the location of all of the files necessary to make this work? Input into lib?  
Why can't we reference our maven repository.

Now the ant file has a dependency on the reverse strategy that has not yet been built.

After a clean the test database needs to be created which requires

does database username in hibernate.cfg.xml have to be in upper case for revenge?  Otherwise connect but don't find any objects.
http://mojo.codehaus.org/maven-hibernate3/hibernate3-maven-plugin/usage.html

The code in the repositories is different than the compiled code in Eclipse.

TODO get rid of /etc entries

building with passwords for the database in the project
Build the mapping files from the database
mvn hibernate3:hbm2hbmxml

hibernate3:hbm2cfgxml
  Generates hibernate.cfg.xml
\begin{tabular}{ l | l}
 
hibernate3:hbm2dao &
  Base class for the different hibernate3 goals based on the Ant tasks of
  hibernate tools. \\ \newline

hibernate3:hbm2ddl &
  Generates database schema \\.

hibernate3:hbm2doc &
  Generates HTML documentation for the database schema. \\

hibernate3:hbm2hbmxml &
  Generates a set of hbm.xml files \\

hibernate3:hbm2java &
  Generates Java classes from set of *.hbm.xml files  \\

hibernate3:hbmtemplate &  Generic exporter that can be controlled by a user provided template or class. \\

hibernate3:help
  Display help information on hibernate3-maven-plugin. Call
    mvn hibernate3:help -Ddetail=true -Dgoal=<goal-name>
  to display parameter details.

\end{tabular}

The package name should be specified as package
  <componentProperties>
            <drop>true</drop>
            <package>org.javautil.mp3.hibernate</package>
            <configurationfile>/src/main/resources/hibernate.cfg.xml</configurationfile>
          </componentProperties>

Not TODO packagename as specified in the documentation.

failure to specify a package name will result in no package.

http://stackoverflow.com/questions/1900234/maven-java-source-code-generation-for-hibernate

\section{Transaction}
Track artists.
Musicians by song.
Found Name vs
\part{Hibernate}
http://mojo.codehaus.org/maven-hibernate3/hibernate3-maven-plugin/componentproperties.html
\begin{tabular}{l | l | l}
 

configurationfile & String & src/main/resources/hibernate.cfg.xml \\ 
propertyfile & String & src/main/resources/database.properties \\
entityresolver & String & org.xml.sax.EntityResolver \\
namingstrategy & String & org.hibernate.cfg.DefaultNamingStrategy \\
persistenceunit & String & \\
packagename & String & \\
revengfile & String & \\
reversestrategy & String & \\
detectmanytomany & boolean & true \\
detectoptmisticlock & boolean & true \\
export & boolean & true \\
update & boolean & false \\
drop & boolean & false \\
create & boolean & true \\
outputfilename & String & \\
delimiter & String & \\
format & boolean & false \\
jdk5 & boolean & false \\
ejb3 & boolean & false \\
filepattern & String & package-name/class-name.ftl \\
template & String & \\

basedir/
exporterclass	String	
scan-classes	boolean	false
\end{tabular} 

should one have a separate class that is an InitializingBean that extends a bean

InitializingBean afterPropertiesSet

eliminate Spring depenendencies
\part{Maven}
\end{document}
